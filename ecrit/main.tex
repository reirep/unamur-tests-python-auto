\documentclass[a4paper]{report}
% Package definition
\usepackage{graphicx}
\usepackage{setspace}
\usepackage{fancybox}

% added package
\usepackage[utf8]{inputenc}
\usepackage[cyr]{aeguill}
\usepackage{xspace}
\usepackage[english,french]{babel}
\usepackage{url}
\usepackage[nottoc,numbib]{tocbibind}
\usepackage{hyperref}


% hacks
\let\urlorig\url
\renewcommand{\url}[1]{%
   \begin{otherlanguage}{english}\urlorig{#1}\end{otherlanguage}%
}

%%%%%%%%%%%%%%%%%%%%%%%%%%%%%%%%%%%%%%%%%%%%%%%%%%%%%%%%%%%%%%%%%%%%%%%%%%%%%%%
%%%%% VARIABLE TO DEFINE %%%%%
\newcommand{\titreMemoire}{Génération de tests unitaires pour simples programmes python}
\newcommand{\auteurMemoire}{Ortegat Pierre}
%%%%%%%%%%%%%%%%%%%%%%%%%%%%%%%

\makeindex

\begin{document}
\thispagestyle{empty}

\begin{center}
\textsc{Universit\'e de Namur}\\
Facult\'e d'informatique\\
Ann\'ee acad\'emique 2021--2022
\end{center}
\vspace{1.3cm}
\hspace{1.4cm}
\fbox{
\begin{minipage}[c][5.4cm]{9.6cm}
\large
\begin{spacing}{1.2}
\begin{center}
\textbf{\titreMemoire}\\
\vspace{1cm}
\auteurMemoire
\end{center}
\end{spacing}
\end{minipage}
}
\vspace{0.5cm}
\begin{figure}[!h]
~~~~~~~~\centering\includegraphics[scale=4.0]{img/unamur.png}
\end{figure}

\normalsize

\vspace{0.5cm}
\begin{table}[!h]
\centering
  \begin{tabular}{ r l }
    Promoteur~: &  \rule{4cm}{0.1mm} {\small (Signature pour approbation du d\'ep\^ot - REE art. 40)}\\
    ~\underline& Xavier Devroey \\\\
    Co-promoteur~: & Benoît Vanderose\\
  \end{tabular}
\end{table}

\vspace{0.5cm}
\begin{center}
M\'emoire pr\'esent\'e en vue de l'obtention du grade de\\
Master en Sciences Informatiques.
\end{center}

\nocite{*}
\chapter{Remerciements}


\chapter{Résumé}

\tableofcontents

\chapter{Introduction}

\chapter{État de l'art}

% intro

\section{Méthodes de test}

%TODO Listing de tous les types de tests dispo et de ce qu'il vallent ; quel sont les avantages / drawbacks

\subsection{A la main}

Le style le plus classique

Prends du temp

biais de confirmation

flackiness

=> techniques automatiques utiles, cqfd

\subsection{Fuzzing}

Quand ca a été inventé, sigification: entrée random dans les progs \cite{Forrester2000}

Ajdh: fort utilisé dans la sécu dès qu'il y a un user input \cite{Godefroid2020} (fait partie du Microsoft Security Development Lifecycle \cite{howard2006security})

\subsubsection{Fuzzing en boite noire}

Prendre tt le prob et donner à l'aveugle des inputs \cite{Forrester2000}

Dépend critiquement d'un set de seed valides à la base si on veut etre efficace

important aussi de limiter le bruit inutile et pas générer plein de shizer

%TODO exemple of blackbox fuzzing

\subsubsection{Fuzzing grammatical ou en boite grise}

greybox fuzzer: \cite{fuzzingbook2022:GreyboxFuzzer}

graybox grammar fuzzer: \cite{fuzzingbook2022:GreyboxGrammarFuzzer}

classic grammar fuzzer:

peach (intégré dans gitlab mtnt) \cite{peach}

spike \cite{spike}

sulley \cite{sulley}

other grammar fuzzer: \cite{sutton2007fuzzing}

g fuzzing pour trouver des failles de sécu dans les browser:  \cite{holler2012fuzzing}

g fuzzing trouver bug complexes dans des compilateurs C  \cite{yang2011finding}

g fuzzing pour trouver les bugs dans les proto réseaux \cite{aflnet}

apprentissage auto gramaire : \cite{bastani2017synthesizing}

tracer process pour créer gramaire automatiquement \cite{hoschele2017mining}


limité par la grammaire en elle même, plus gros défaut

%TODO exemple de greybox fuzzing : fuzzing par ligne ? 

\subsubsection{Fuzzing en boite blanche et exécution symbolique}

parser le prog, le faire tourner et tenter de résoudre les conditions pour toucher toutes les branches avec un solveur.

Plus efficace pour un covering complet et pour taper sur toutes les branches et chopper les bugs de meeeeeeeeeeeeeeeeeeeeeeeeeeeeeeeeeeeeeeeee


dynamic execution testing: SAGE \cite{godefroid2008automated} (symbolic execution x86 level avec opti pour enorme stack traces \cite{Godefroid2020} ) 

qui étends le travail d'autre sur le génération de tests auto \cite{Godefroid2020} \cite{cadar2005execution} \cite{godefroid2005dart}

utilisé en prod partout, plus de 100 année machines dans les dents
"largest computational usage ever for any Satisfiability-Modulo-Theories (SMT) solver" d'après les auteurs de z3 \cite{moura2008z3}

\subsection{Property based}



\cite{Fink1997}

\cite{Paraskevopoulou2015}

\cite{Papadakis2011}


\subsection{Fault injection}




\subsection{Utilisation conjointe}

NOTE: all can be combined to try to be more efficient !! \cite{Godefroid2020} => Hybrid fuzzing

plusieurs approches en meme temps: Portfolio approaches.

\chapter{Développement}

\section{Sélection de stratégie de tests}

% pourquoi est ce que les tests sélectionnées ont été sélectionés

\subsection{Innovations}

% combinaison

% "hot" lines


\section{Compromis}

% certains détail d'impl qui sont utiles à savoir

\subsection{Combinaisons de techniques}
% fuzzing into combiner

\subsection{Complexité spatiale vs temporelle}
% space complexity vs time complexity

\section{Efficacité \& limitations}

\subsection{Résolution des branches}

\subsubsection{Le problème d'arrêt}

\subsection{Complexité temporelle}

\subsection{Valeurs par défaut}

\section{Intégration}

\subsection{Interface unifiée}

\subsection{Gestion des erreurs}

\subsection{Intégration dans Inginious}



\chapter{Conclusion}

\bibliographystyle{plain}
\bibliography{ref} 

\chapter{Annexes}



\end{document}
